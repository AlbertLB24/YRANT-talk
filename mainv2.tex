\documentclass{beamer}
\usepackage{amsmath}
\usepackage{xcolor}
\usepackage{multimedia}
\usetheme{copenhagen}
\definecolor{purple}{rgb}{0.3,0,0.4}
\definecolor{aqua}{rgb}{0,0.85,0.8}
\definecolor{grey}{rgb}{60,60,60}
\setbeamercolor*{palette primary}{use=structure, fg=white, bg=purple}
\setbeamercolor*{background canvas}{bg=purple!5}
\setbeamercolor*{block title example}{use=structure,fg=white,bg=purple}
\setbeamercolor*{block body example}{fg=black,use=block title,bg=purple!20}
\setbeamercolor*{block title}{use=example text,fg=white,bg=aqua!60!black}
\setbeamercolor*{block body}{fg=black,use=block title example,bg=aqua!10}

\usepackage{graphicx}
\usepackage{tikz-cd}

\newcommand{\Gal}{\mathrm{Gal}}
\newcommand{\Cl}{\mathrm{Cl}}
\newcommand{\Rep}{\mathrm{Rep}}
\newcommand{\Irr}{\mathrm{Irr}}
\newcommand{\Ind}{\mathrm{Ind}}
\newcommand{\GL}{\mathrm{GL}}
\newcommand{\PGL}{\mathrm{PGL}}
\newcommand{\SL}{\mathrm{SL}}
\newcommand{\Sp}{\mathrm{Sp}}
\newcommand{\End}{\mathrm{End}}
\newcommand{\cInd}{c-\mathrm{Ind}}
\newcommand{\ind}{\mathrm{ind}}
\newcommand{\aff}{\mathrm{aff}}

\newcommand{\CC}{\mathbb{C}}
\newcommand{\FF}{\mathbb{F}}
\newcommand{\NN}{\mathbb{N}}
\newcommand{\PP}{\mathfrak{P}}
\newcommand{\QQ}{\mathbb{Q}}
\newcommand{\RR}{\mathbb{R}}
\newcommand{\ZZ}{\mathbb{Z}}
\newcommand{\GG}{\mathbb{G}}
\newcommand{\adele}{\mathbb{A}}
\newcommand{\pp}{\mathfrak{p}}
\newcommand{\qq}{\mathfrak{q}}
\newcommand{\rr}{\mathfrak{r}}
\newcommand{\af}{\mathfrak{a}}
\newcommand{\cH}{\mathcal{H}}


\theoremstyle{plain}
\newtheorem{thm}{Theorem}[section]
\newtheorem{rem}[thm]{Remark}
\newtheorem{proposition}[thm]{Proposition}



\graphicspath{ }


\title{Parameters of Hecke algebras for p-adic groups}
\author{Albert Lopez Bruch}
%\institute{LSGNT}
\date{5 September, 2025}

\begin{document}

\frame{\titlepage}


\begin{frame}
    \frametitle{Introduction}
    \textbf{Setup:} Let $F$ be a non-archimedean local field with residue field $\mathbb{F}_q$ (think of $F$ as a finite extension of $\QQ_p$), and let $G$ be a connected reductive group over $F$.
    \vspace{0.3cm}

    For example, think of $G$ as $\GL_n(F), \SL_n(F), \Sp_{2n}(F)\ldots$, but also as an exceptional group such as $G_2(F)$.
    \vspace{0.3cm}

    We will denote by $\Rep(G)$ the category of smooth admissible complex representations of $G$.

    \begin{fact}[supercuspidal representations as building blocks]
        For any irreducible object $(\pi,V)$ in $\Rep(G)$, there is some parabolic subgroup $P\subseteq G$ with Levi subgroup $M$ and supercuspidal representation $\sigma$ of $M$ such that $\pi\hookrightarrow\Ind_P^G\sigma$.
    \end{fact}
\end{frame}

\begin{frame}
    \frametitle{Bernstein Decomposition}
    \begin{theorem}[Bernstein]
        There is a direct product decomposition
        \[\Rep(G)\cong \prod_{[M,\sigma]\in\mathfrak{J}(G)}\Rep(G)_{[M,\sigma]}\]
        into full indecomposable categories $\Rep(G)_{[M,\sigma]}$ known as \textit{Bernstein blocks}. The product ranges over conjugacy classes of pairs $(M,\sigma)$, denoted by $[M,\sigma]\in\mathfrak{J}(G)$.
    \end{theorem}
    \textbf{Example:} $\Rep(G)_{[T,\mathbf{1}]}$ is the \textit{principal block}.
    \vspace{0.3cm}

    \textbf{Upshot:} We study the irreducible objects $\Irr(G)_{[M,\sigma]}$ of each block individually, and the extensions between them.
\end{frame}


\begin{frame}
    \frametitle{Hecke Algebras}
    Consider a pair $(K,\rho)$ where
    \begin{itemize}
        \item $K$ is a compact open subgroup of $G$.
        \item $(\rho,W)$ is a smooth irreducible representation of $K$.
    \end{itemize}
    For such a pair, we construct the associated Hecke algebra
    \[\cH(G,K,\rho):=\End_G(\ind_K^G\rho)\]

    \textbf{Example:} If $\rho=\textbf{1}$ is the trivial character, then 
    \[\cH(G,K,\rho)=C_c(K\backslash G/K)\]
    is the space of locally constant, compactly supported and $K$-invariant complex functions on $G$, equipped with the convolution product.
    
    
\end{frame}



\begin{frame}
    \frametitle{Theory of types}
    To study each block using Hecke algebras, we use the theory of types introduced by Bushnell-Kutzko.

    \begin{theorem}[Kim, Yu, Fintzen, Kaletha, Spice]
        Under mild conditions, one can associate to any $[M,\sigma]\in\mathfrak{J}(G)$ a pair $(K,\rho)$ (called a $[M,\sigma]$-type) such that there is an equivalence of categories
        \[ \Rep(G)_{[M,\sigma]}\cong\text{ right }\cH(G,K,\rho)-\text{modules}.\]
    \end{theorem}

    \textbf{Example:} The pair $(\text{I},\mathbf{1})$ is a $[T,\mathbf{1}]$-type. Thus, 
    \begin{align*}
        \Rep(G)_{[T,\mathbf{1}]}&\cong\text{ right }C_c(I\backslash G/I)-\text{modules}\\ 
        &\cong\{(\pi,V)\in\Rep(G):V\text{ is generated by }V^I\}
    \end{align*}
    \textbf{Main idea:} Hecke algebras reduce infinite dimensional problems to finite-dimensional ones.
   
\end{frame}

\iffalse
\begin{frame}
    \frametitle{Examples}
    \textbf{Example:} $\cH(G,K,\mathbf{1})\cong C_c(K\backslash G/K)$ with the convolution product. There is a bijection
    \[\{(\pi,V)\in\Irr(G):V^K\neq 0\}\longleftrightarrow \text{ irreducible }C_c(K\backslash G/K) \text{-mod.}\]
    \vspace{0.5cm}
\end{frame}
\fi

\begin{frame}
    \frametitle{Questions}
    All these results are useful if we
    \begin{enumerate}
        \item understand structure of Hecke algebras,
        \item describe their irreducible modules.
    \end{enumerate}
    \vspace{0.5cm}
    \textbf{Example:} Suppose that $G$ is semisimple and that $\pi\in\Irr(G)$ is supercuspidal of depth-zero. Then $\pi=\ind_K^G\rho$ for some pair $(K,\rho)$, and this is a $[G,\pi]$-type! Thus,
    \[\cH(G,K,\rho)=\End_G(\pi)=\CC,\]
    so $\pi$ is the only irreducible element of $\Rep(G)_{[G,\pi]}$, and it has no nontrivial extensions!
\end{frame}


\begin{frame}
    \frametitle{The Iwahori-sperical Hecke algebra}
    Let $G$ be a semisimple split adjoint group (e.g. $G=\PGL_n(F)$) with maximal torus $T$. Associated to $G$, there is the extended affine Weyl group $\widetilde{W}=N_G(T)(F)/T(\mathcal{O}_F)$ with the properties:
    \begin{enumerate}
        \item There is a canonical $\CC$-basis $\{T_w:w\in\widetilde{W}\}$ of $\cH(G,I,\mathbf{1})$.
        \item There is a semidirect product $\widetilde{W}=W_{\aff}\rtimes\Omega$ with $\Omega$ finite group and $(W_{\aff},S_{\aff})$ an affine Coxeter group.
        \item This decomposition induces an isomorphism of $\CC$-algebras
        \[\cH(G,I,\mathbf{1})=\cH(W_{\aff},S_{\aff},q)\ \tilde{\otimes}\ \CC[\Omega].\]
        \item $\cH(W_{\aff},S_{\aff},q)$ has basis $\{T_w:w\in W_{\aff}\}$ and relations
        \begin{itemize}
            \item $T_{w_1}T_{w_2}=T_{w_1w_2}$ if $l(w_1w_2)=l(w_1)+l(w_2)$,
            \item $T_s^2=(q-1)T_s+qT_1$ if $s\in S_{\aff}$.
        \end{itemize}
    \end{enumerate}
\end{frame}

\begin{frame}
    \frametitle{Hecke algebras in general}
    Similar results hold for general types $(K,\rho)$. 
    \begin{enumerate}
        \item There is a group $W(\rho)$ and a canonical $\CC$-basis $\{T_w:w\in W(\rho)\}$ of $\cH(G,K,\rho)$.
        \item There is a semidirect product $W(\rho)=W(\rho)_{\aff}\rtimes\Omega(\rho)$ with $(W(\rho)_{\aff},S(\rho)_{\aff})$ an affine Coxeter group.
        \item This decomposition induces an isomorphism of $\CC$-algebras
        \[\cH(G,K,\rho)=\cH(W_{\aff}(\rho),S_{\aff}(\rho),q)\ \tilde{\otimes}\ \CC[\Omega],\]
        where $q:S(\rho)_{\aff}\rightarrow\QQ_{>1}$ is a parameter function.
    \end{enumerate}
    \vspace{0.3cm}
    \textbf{Open problem:} Determine the parameter function $q:S(\rho)_{\aff}\rightarrow\QQ_{>1}$ in the modular representation setting. 
\end{frame}



\begin{frame}
    Thank you for listening!
\end{frame}

\end{document}