\documentclass{beamer}
\usepackage{amsmath}
\usepackage{xcolor}
\usepackage{multimedia}
\usetheme{copenhagen}
\definecolor{purple}{rgb}{0.3,0,0.4}
\definecolor{aqua}{rgb}{0,0.85,0.8}
\definecolor{grey}{rgb}{60,60,60}
\setbeamercolor*{palette primary}{use=structure, fg=white, bg=purple}
\setbeamercolor*{background canvas}{bg=purple!5}
\setbeamercolor*{block title example}{use=structure,fg=white,bg=purple}
\setbeamercolor*{block body example}{fg=black,use=block title,bg=purple!20}
\setbeamercolor*{block title}{use=example text,fg=white,bg=aqua!60!black}
\setbeamercolor*{block body}{fg=black,use=block title example,bg=aqua!10}

\usepackage{graphicx}
\usepackage{tikz-cd}

\newcommand{\Gal}{\mathrm{Gal}}
\newcommand{\Cl}{\mathrm{Cl}}
\newcommand{\Rep}{\mathrm{Rep}}
\newcommand{\Irr}{\mathrm{Irr}}
\newcommand{\Ind}{\mathrm{Ind}}
\newcommand{\GL}{\mathrm{GL}}
\newcommand{\SL}{\mathrm{SL}}
\newcommand{\Sp}{\mathrm{Sp}}
\newcommand{\End}{\mathrm{End}}
\newcommand{\cInd}{c-\mathrm{Ind}}
\newcommand{\ind}{\mathrm{ind}}

\newcommand{\CC}{\mathbb{C}}
\newcommand{\FF}{\mathbb{F}}
\newcommand{\NN}{\mathbb{N}}
\newcommand{\PP}{\mathfrak{P}}
\newcommand{\QQ}{\mathbb{Q}}
\newcommand{\RR}{\mathbb{R}}
\newcommand{\ZZ}{\mathbb{Z}}
\newcommand{\GG}{\mathbb{G}}
\newcommand{\adele}{\mathbb{A}}
\newcommand{\pp}{\mathfrak{p}}
\newcommand{\qq}{\mathfrak{q}}
\newcommand{\rr}{\mathfrak{r}}
\newcommand{\af}{\mathfrak{a}}
\newcommand{\cH}{\mathcal{H}}

\theoremstyle{plain}
\newtheorem{thm}{Theorem}[section]
\newtheorem{rem}[thm]{Remark}
\newtheorem{proposition}[thm]{Proposition}



\graphicspath{ }


\title{Parameters of Hecke algebras for p-adic groups}
\author{Albert Lopez Bruch}
%\institute{LSGNT}
\date{5 September, 2025}

\begin{document}

\frame{\titlepage}


\begin{frame}
    \frametitle{Introduction}
    \textbf{Setup:} Let $F$ be a non-archimedean local field with residue field $\mathbb{F}_q$ (think of $F$ as a finite extension of $\QQ_p$), and let $G$ be a connected reductive group over $F$.
    \vspace{0.3cm}

    For example, think of $G$ as $\GL_n(F), \SL_n(F), \Sp_{2n}(F)\ldots$, but also as an exceptional group such as $G_2(F)$.
    \vspace{0.3cm}

    We will denote by $\Rep(G)$ the category of smooth admissible complex representations of $G$.

    \begin{fact}[supercuspidal representations as building blocks]
        For any irreducible object $(\pi,V)$ in $\Rep(G)$, there is some parabolic subgroup $P\subseteq G$ with Levi subgroup $M$ and supercuspidal representation $\sigma$ of $M$ such that $\pi\hookrightarrow\Ind_P^G\sigma$.
    \end{fact}
\end{frame}

\begin{frame}
    \frametitle{Bernstein Decomposition}
    \begin{theorem}[Bernstein]
        There is a direct product decomposition
        \[\Rep(G)\cong \prod_{[M,\sigma]\in\mathfrak{J}(G)}\Rep(G)_{[M,\sigma]}\]
        into full indecomposable categories $\Rep(G)_{[M,\sigma]}$ known as \textit{Bernstein blocks}. The product ranges over conjugacy classes of pairs $(M,\sigma)$, denoted by $[M,\sigma]\in\mathfrak{J}(G)$.
    \end{theorem}
    \textbf{Example:} $\Rep(G)_{[T,\mathbf{1}]}$ is the \textit{principal block}.
    \vspace{0.3cm}

    \textbf{Upshot:} We study the irreducible objects $\Irr(G)_{[M,\sigma]}$ of each block individually, and the extensions between them.
\end{frame}



\begin{frame}
    \frametitle{Theory of types}
    Bushnell and Kutzko introduced a rich theory to study each block individually.

    \begin{definition}[Types]
        Let $[M,\sigma]\in\mathfrak{J}(G)$. A pair $(K,\rho)$ is a $[M,\sigma]$-type if for any $(\pi,V)\in\Irr(G)$,
        \[(\pi,V)\in\Rep(G)_{[M,\sigma]}\iff \pi|_K \text{ contains }\rho.\]
    \end{definition}
    \textbf{Example:} The pair $(\text{I},\mathbf{1})$ is a $[T,\mathbf{1}]$-type. Thus, 
    \[\Rep(G)_{[T,\mathbf{1}]}=\{(\pi,V)\in\Rep(G):V\text{ is generated by }V^I\}\]
    %$(\pi,V)\in\Rep(G)$ lies in the principal block if and only if $V^I$ generate $V$ as a $G$-rep!
   
\end{frame}


\begin{frame}
    \frametitle{Hecke Algebras}
    For a pair $(K,\rho)$, we construct the associated Hecke algebra
    \[\cH(G,K,\rho):=\End_G(\ind_K^G\rho)\]
    \begin{theorem}
        Suppose that $(K,\rho)$ is an $[M,\sigma]$-type. Then there is an equivalence of categories 
        \begin{align*}
            \Rep(G)_{[M,\sigma]}&\longrightarrow\text{ right }\cH(G,K,\rho)-\text{modules}\\
            (\pi,V)&\longmapsto Hom_K(V,W)=Hom_G(\ind_K^G V,W)
        \end{align*}
    \end{theorem}
    \textbf{Main idea:} Hecke algebras reduce infinite dimensional problems to finite-dimensional ones.

\end{frame}

\begin{frame}
    \frametitle{Examples}
    \textbf{Example:} $\cH(G,K,\mathbf{1})\cong C_c(K\backslash G/K)$ with the convolution product. There is a bijection
    \[\{(\pi,V)\in\Irr(G):V^K\neq 0\}\longleftrightarrow \text{ irreducible }C_c(K\backslash G/K) \text{-mod.}\]
    \vspace{0.5cm}

    \textbf{Advanced example:} Suppose that $G$ is semisimple and that $\pi\in\Irr(G)$ is supercuspidal of depth-zero. Then $\pi=\ind_K^G\rho$ for some pair $(K,\rho)$, and this is a $[G,\pi]$-type! Thus,
    \[\cH(G,K,\rho)=\End_G(\pi)=\CC,\]
    so $\pi$ is the only irreducible element of $\Rep(G)_{[G,\pi]}$, and it has no nontrivial extensions!
\end{frame}

\begin{frame}
    \frametitle{Questions}
    All these results are useful if we
    \begin{enumerate}
        \item know that types exist for the Bernstein blocks,
        \item understand structure of Hecke algebras,
        \item describe their irreducible modules.
    \end{enumerate}
    The work of Kim and Yu, and Fintzen, Kaletha and Spice show that under mild conditions types can be constructed for all blocks.
\end{frame}


\begin{frame}
    \frametitle{The Iwahori-Hecke algebra}
\end{frame}


\begin{frame}
    Thank you for listening!
\end{frame}

\end{document}